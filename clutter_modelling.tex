\documentclass[12pt,twoside]{paper}
\usepackage[left=3.0cm, right = 2.5cm,bottom=3.5cm,headsep=12pt]{geometry}

%include necessary packages      
\usepackage{graphicx}
\usepackage{amsmath}
\usepackage{float}
\usepackage{supertabular}
\usepackage[section]{placeins}
\usepackage{array}


\begin{document}
\section{Introduction}
The K-distribution is well used to model the amplitude statistics of sea clutter and has been repeatedly demonstrated to be an appropriate empirical fit to measured data \cite{ward_sea_clutter}. TBD ...

\section{Statistical Modeling}

\subsection{Infinite Number of Scatterers}

\subsubsection{Case 1: Only Random Scatterers}

\subsubsection{Case 2: Random Scatterers with a Deterministic Component}

\subsubsection{Case 3: Random Scatterers of Dissimilar Sizes}
???

\subsection{Finite Number of Scatterers}
\subsubsection{Case 4: Fluctuating Mean}

2-D random walk in the complex plane

characteristic function

let N go to infinity

add short orbit path

show RMT relation

\cite{jakeman_significance} \cite{jakeman_generalized} \cite{jakeman_model_non_rayleigh}.

\cite{yeh_fading} \cite{yeh_first_principles}

\cite{hart_so} \cite{yeh_universal}

\cite{zheng_single}


\bibliographystyle{unsrt}     
\bibliography{clutter}
\end{document}